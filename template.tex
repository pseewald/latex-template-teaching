% This is a template presentation for computer chemistry class UZH Spring Semester 2017

% change font size if needed:
\documentclass[xcolor={dvipsnames},10pt]{beamer}

% some definitions / defaults:
\newcommand{\topic}[1]{\newcommand{\inserttopic}{#1}}
\defbeamertemplate{description item}{align left}{\insertdescriptionitem\hfill}

\mode<presentation>
{
    % themes and formatting specs:
    \useinnertheme{circles}
    \useoutertheme{default}
    \usefonttheme{structurebold}
    \hypersetup{
        pdfpagelayout=OneColumn,
        pdfview=Fit,
        pdfstartview=Fit
    }
    \beamertemplatenavigationsymbolsempty
    \setbeamertemplate{description item}[align left]

    % format of frametitles
    \setbeamertemplate{frametitle}
    {
        \vskip 7pt
        \begin{beamercolorbox}[wd=\paperwidth]{frametitle}%
            \begin{centering}
                \usebeamerfont{frametitle}\insertframetitle\par
                \usebeamerfont{framesubtitle}\insertframesubtitle\par
                \rule[4mm]{\paperwidth}{1.5pt}
            \end{centering}
        \end{beamercolorbox}
    }

    % remove space between frametitle and content
    \addtobeamertemplate{frametitle}{}{\vspace{-1cm}}

    % format and content of title page
    \setbeamertemplate{title page}
    {
        \begin{beamercolorbox}[wd=\paperwidth,dp=2cm]{title}%
            \begin{centering}
                \rule[0mm]{\paperwidth}{1.5pt}
                \usebeamerfont{title}
                Computer Chemistry Class\par
                \usebeamerfont{subtitle}
                \insertsubtitle\par
                \vspace{0.5cm}
                {\footnotesize
                    Spring Semester 2017\par
                    \insertauthor\par
                    \rule[2mm]{\paperwidth}{1.5pt}
                }
                {
                    \usebeamercolor[fg]{normal text}
                    \textnormal
                    \inserttopic\par
                }
            \end{centering}
        \end{beamercolorbox}
    }

    % display frame numbers
    \setbeamertemplate{footline}[frame number]

    % how to display covered stuff
    \setbeamercovered{invisible}

    % colors and fonts
    \setbeamercolor{structure}{bg=CadetBlue!30, fg=OliveGreen}
    \setbeamercolor{block title}{use={structure},bg=structure.bg, fg=structure.fg}
    \setbeamercolor{block title alerted}{use={structure},bg=structure.bg}
    \setbeamercolor{block title example}{use={structure},fg=BlueViolet, bg=structure.bg}
    \setbeamercolor{background canvas}{use={structure}, bg=structure.bg!20}
    \setbeamercolor{titlelike}{use={background canvas},bg=background canvas.bg}
    \setbeamercolor{block body}{use={structure},bg=structure.bg!50}
    \setbeamercolor{block body alerted}{use={structure},bg=structure.bg!50}
    \setbeamercolor{block body example}{use={structure},bg=structure.bg!50}
    \setbeamercolor{alerted text}{fg=Orange}
    \setbeamerfont{framesubtitle}{size=\small}
}

% packages to use:
\usepackage[english]{babel}
\usepackage{booktabs}
\usepackage[latin1]{inputenc}
\usepackage{times}
\usepackage[T1]{fontenc}
\usepackage{framed}
\usepackage[font=small]{caption}

% Title page:
\subtitle{Tutorial day X}
\topic{Geometry optimization\par
    topic 2\par
    topic 3\par
topic 4}
\author{Your~name}

% If you wish to uncover everything in a step-wise fashion, uncomment
% the following command:
%\beamerdefaultoverlayspecification{<+->}
\begin{document}

\begin{frame}[noframenumbering, plain]
    \titlepage
\end{frame}

\begin{frame}{Title}{Optional Subtitle}
    content
\end{frame}

\begin{frame}{Lists}
    \begin{columns}[T,onlytextwidth]
        \column{0.33\textwidth}
        Items
        \begin{itemize}
            \item something \item something else
        \end{itemize}

        \column{0.33\textwidth}
        Enumerations
        \begin{enumerate}
            \item First, \item Second and \item Last.
        \end{enumerate}

        \column{0.33\textwidth}
        Descriptions
        \begin{description}
            \item[Gaussian] A code \item[GaussView] A GUI
        \end{description}
    \end{columns}
\end{frame}

\begin{frame}{Highlighting text}
    Different ways to highlight text
    \begin{itemize}
        \item
            \emph{emphasized} text
        \item
            \alert{alerted} text
        \item
            \textbf{bold} text
        \item
            \texttt{typewriter}
    \end{itemize}
\end{frame}

\begin{frame}{Math}
    \begin{align*}
        L' = {L}{\sqrt{1-\frac{v^2}{c^2}}}
    \end{align*}
\end{frame}

\begin{frame}{Tables}

    \begin{table}
        \begin{tabular}{l*{6}{c}r}
            \toprule
            Team              & P & W & D & L & F  & A & Pts \\
            \midrule
            Manchester United & 6 & 4 & 0 & 2 & 10 & 5 & 12  \\
            Celtic            & 6 & 3 & 0 & 3 &  8 & 9 &  9  \\
            Benfica           & 6 & 2 & 1 & 3 &  7 & 8 &  7  \\
            FC Copenhagen     & 6 & 2 & 1 & 3 &  5 & 8 &  7  \\
            \bottomrule
        \end{tabular}
    \end{table}

\end{frame}

\begin{frame}{Figures}
    \begin{figure}[ht!]
        \centering
        \includegraphics[width=\textwidth]{cat.jpg}
        \caption{dead or alive?}
    \end{figure}
\end{frame}

\begin{frame}{Columns}
    \begin{columns}[T,onlytextwidth]
        \column{.5\textwidth}
        Contents of the first column
        \column{.5\textwidth}
        Contents split \\ into two lines
    \end{columns}
\end{frame}

\begin{frame}{Blocks}

    \begin{block}{This is a Block}
        This is important information
    \end{block}

    \begin{alertblock}{This is an Alert block}
        This is an important alert
    \end{alertblock}

    \begin{exampleblock}{This is an Example block}
        This is an example
    \end{exampleblock}

\end{frame}

\begin{frame}{Boxes}
    \fcolorbox{black}{white}{\parbox[b][6em][t]{0.2\textwidth}{Some \\ text} }
    \fcolorbox{black}{white}{\parbox[c][5em][s]{0.3\textwidth}{Some \vfill other text} }
    \fcolorbox{black}{white}{\parbox[t][4em][c]{0.4\textwidth}{Some \\ more text} }
\end{frame}

\begin{frame}[allowframebreaks]
    \frametitle<presentation>{Bibliography}
    \begin{thebibliography}{10}
        \setbeamertemplate{bibliography item}[book]
        \bibitem{Autor1990}
        A.~Autor.
        \newblock {\em Introduction to Giving Presentations}.
        \newblock Klein-Verlag, 1990.
        \setbeamertemplate{bibliography item}[article]
        \bibitem{Jemand2000}
        S.~Jemand.
        \newblock On this and that.
        \newblock {\em Journal of This and That}, 2(1):50--100, 2000.
    \end{thebibliography}
\end{frame}

\begin{frame}{Overlays}
    You can create overlays\dots
    \begin{itemize}
        \item using the \texttt{pause} command:
            \begin{itemize}
                \item
                    First item.
                    \pause
                \item
                    Second item.
            \end{itemize}
        \item
            using overlay specifications:
            \begin{itemize}
                \item<3->
                    First item.
                \item<4->
                    Second item.
            \end{itemize}
        \item
            using the general \texttt{uncover} command:
            \begin{itemize}
                \uncover<5->{\item
                First item.}
                \uncover<6->{\item
                Second item.}
        \end{itemize}
    \end{itemize}
\end{frame}
\end{document}
